\documentclass[12pt, twocolumn]{article}
\usepackage{amsfonts}
\usepackage{amsmath}
\usepackage{amssymb}
\usepackage{geometry}
\usepackage{microtype}
\usepackage{titlesec}
\usepackage[utf8]{inputenc}
\usepackage[spanish]{babel}
\usepackage[T1]{fontenc}
\usepackage{graphicx}
\usepackage{hyperref}

\title{}
\geometry {
    left=1.59cm,
    right=1.58cm,
    tmargin=1.9cm,
    bottom=2.534cm
}
\begin{document}
% TITLEPAGE
%///////////////////////////////////////////////////////////////////////////////////////////////////////////////////////////
\begin{titlepage}
\centering
{\includegraphics[width=0.2\textwidth]{../../utnlogo.png}\par}
\vspace{1cm}
{\bfseries\LARGE Ingeniería en Sistemas de Información\par}
\vspace{0.5cm}
{\bfseries\LARGE Universidad Tecnológica Nacional\par}
\vspace{1cm}
{\scshape\Large Facultad Regional Resistencia \par}
\vspace{1cm}
{\scshape\Huge Análisis Numérico \par}
\vspace{1cm}
{\itshape\Large Previa 1: Conceptos de Complejos\par}
\vspace{1cm}
\vfill
{\Large Profesores:\par}
{\Large Ing. García Claudia, Roxana \par}
{\Large Nápoles Valdés, Juan \par}
\vfill
{\Large Alumnos:\par}
{\Large Acosta Quintana, Lautaro \par}
{\Large Guzmán, Tomás \par}
{\Large Rosín, Zaira \par}
{\Large Stegmayer, Tobías \par}
\vfill
{\Large Agosto 2023\par}
\end{titlepage}
%///////////////////////////////////////////////////////////////////////////////////////////////////////////////////////////

\textbf{\Large Números Complejos}
\begin{enumerate}
    \item \textbf{Definición}
        \begin{itemize}
            \item Par ordenado de números reales de la forma genérica: $\boxed{Z=(a,b)/a,b\in\mathbb{R}}$.
            \item $a$ represente al componente real de Z($Re(Z)$) y $b$ representa al componente imaginario de $Z$ ($Im(Z)$).
            \item Pueden ser representados en un plano como un punto o el vector que tiene como extremo a dicho punto.
        \end{itemize}
    \item \textbf{Suma de Complejos}
        \begin{itemize}
            \item Se suman entre sí las partes reales y las partes imaginarias, tal como se hace con los vectores:
                $$ \boxed{(a,b) + (c,d)\underset{Df}{=}(a+c,b+d) }$$
        \end{itemize}
    \item \textbf{Multiplicación de Complejos} 
        $$\boxed{(a,b)\cdot(c,d) \underset{Df}{=} (a\cdot c-b\cdot d,a\cdot d + b\cdot c)}$$
    \item \textbf{Complejos de la forma $(a,0)=a$}
    \item \textbf{Complejo $\boxed{i\underset{Df}{=}(0,1)}$}
        \begin{itemize}
            \item Si tenemos un número complejo cualquiera $(a,b)$, es posible representarlo de la siguiente manera:
                $$(a,b)=(a,0)+(0,b)=\boxed{a+bi}$$
            \item $i^{2}=i\cdot i = (0,1)\cdot (0,1) = (-1,0) = -1 $
            \item \textbf{La multiplicación de Complejos} podrá escribirse como:
                \begin{equation*}
                    \begin{aligned}
        & (a+bi)\cdot (c+di) = ac+adi + bci + bdi^{2}\\
        & =\boxed{ac-bd+(ad+bc)i}
                    \end{aligned}
                \end{equation*}
        \end{itemize}
    \item \textbf{Notación Polar o Trigonométrica}
        \begin{itemize}
            \item El ángulo comprendido entre el eje real positivo del plano complejo y la línea que une $Z$ con el origen de dicho plano se denomina \textbf{argumento del complejo $\alpha$}.
            \item $arg \ Z=\alpha \pm 2\pi k$
            \item \textbf{Argumento principal} es aquel que varía entre $\pi$ y $-\pi$: $$Arg \ Z=\alpha \pm 2\pi k \Leftrightarrow -\pi\leq\alpha\leq\pi$$
            \item El complejo $(a,b)$ puede ser escrito en forma:
$(a,b)=a+bi=\boxed{r\cos{\alpha}+r\sin{\alpha}i}$
        \end{itemize}
    \item \textbf{Notación Exponencial}
        \begin{itemize}
            \item $ei^{\theta}=\cos{\theta}+\sin{\theta}i \Rightarrow\\
            (a,b)=a+bi =r\cos{\theta}+r\sin{\theta}i=
                        r(\cos{\theta}+\sin{\theta}i)=
                   \boxed{re^{i\theta}}$
        \end{itemize}
   \item \textbf{Bibliografía}:
\begin{itemize}
    \item \href{https://www.youtube.com/watch?v=5FemcGdN3Xw}{\small El Traductor de Ingeniería. (23 de mayo de 2017). NÚMEROS COMPLEJOS: Lic. María Inés Baragatti - Parte 1 | Docentes Apasionadxs 2017 [Archivo de Vídeo]. Youtube.}
    \item \href{https://www.youtube.com/watch?v=f7iOdIaourk}{\small El Traductor de Ingeniería. (24 de mayo de 2017). NÚMEROS COMPLEJOS: Lic. María Inés Baragatti - Parte 2 | Docentes Apasionadxs 2017 [Archivo de Vídeo]. Youtube.}
\end{itemize}
\end{enumerate}
\end{document}

