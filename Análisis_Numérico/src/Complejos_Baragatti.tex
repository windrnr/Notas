# Definición
Un número complejo es un par ordenado de números reales de la forma genérica: $Z=(a,b)/a,b \in \mathbb{R}$. Ej.: $Z_1=(3,-2).$
A diferencia de un par ordenado común, en donde $a$ y $b$ representan la abscisa y la ordenada, un número complejo $Z$ es un par ordenado en donde $a$ representa el componente real de $Z$ ($Re(Z)$)y $b$ representa al componente imaginario de $Z$ ($Im(Z)$).

# Gráfico
Para graficar un número complejo según su **par ordenado**, se marca en las abscisas la parte real, y en las ordenadas la parte imaginaria. Entonces, puede ser representado como el punto en el plano con esas coordenadas o como el vector que tiene como extremo a dicho punto.

# Suma de complejos
La suma de complejos se realiza sumando entre sí las partes reales y las partes imaginarias de cada número complejo:
$$ (a,b) + (c,d)\underset{Df}{=}(a+c,b+d) $$

# Multiplicación de complejos
El producto entre dos números complejos se obtiene de la siguiente manera:
1. Para obtener la parte real: Se realiza una multiplicación entre las partes reales menos la multiplicación de las partes imaginarias de ambos números
2. Para obtener la parte imaginaria: Se realiza una multiplicación de la parte real del primer número con la parte imaginaria del segundo más la multiplicación de la parte imaginaria del primero con la parte real del segundo.
$$(a,b)\cdot(c,d) \underset{Df}{=} (a\cdot c-b\cdot d,a\cdot d + b\cdot c) $$
> **Regla mnemotécnica:** "primera por primera menos segunda por segunda y primera por segunda más segunda por primera"



